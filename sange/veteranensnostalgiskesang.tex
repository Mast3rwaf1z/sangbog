\begin{sang}{Veteranens nostalgiske sang}{Melodi: Den gamle pavillon \ldots}
\spal2
\begin{vers}
Jeg er cand.scient. fra Aalborg UC 
og jeg ska' nu til 
at prøv' at brug' det 
Ved `Christian Rovsing' er jeg blevet ansat 
Min frihed pantsat 
i Ballerup 
Men når jeg er her i mit gamle hus 
vejrer jeg jo fordums sus 
Jeg drømmer mig tilbage til det gamle institut 
til de kurser som vi nød 
til lærerpræk ved tavlen hvor vi talte hvert minut, 
kaffestuens fredagsbrød 
En pause kunne strækkes meget længe, 
for kaf'maskin' var ganske uden fut 
Jo, vi forstod i kaffestuen at hænge 
på det gamle institut
\end{vers}
\begin{vers}
Tænker på de fine F-klub-fester 
Ludo-kontester 
- nu er vi gæster 
Når jeg kommer her er jeg en fremmed, 
har jeg fornemmet, 
det er lidt hårdt 
For det er jo her at mit hjerte slår 
gennem mange skønne år 
Jeg drømmer mig tilbage til det gamle institut 
Grupperummet hvor vi sad 
Med pæne diskussioner uden gnist og uden krudt 
Kors i røven hvor vi gad 
Men samfundet er fuldt af direktioner  
der sørger for at freden den bli'r brudt 
mens der var rare mænd og ditto koner  
på det gamle institut
\end{vers}
\laps
\end{sang}