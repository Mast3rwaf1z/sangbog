\begin{sang}{Hjemmebrænderiet}{Melodi - Trad.}
\spal2
\begin{vers}
Jeg bor her i Stenhøj i't skævt bette hus,
alene rent bortset fra rotter og mus,
i gården et lokum og bagved et skur,
og i det står mit brændevinsapparatur.
{\em Det syder af fusel og bobler af gær,
din lever tar skade og øjet får stær,
det smager som rævepis og terpentin,
men det er billligt og så bli'r man fuld som et svin.}
\end{vers}
\begin{vers}
Min bedstemor brændte sin brændevin selv,
det slog både hende og manden ihjel. 
De drak aldrig selv, men for heden den dag,
brænderiet røg i luften med et ordentligt brag.
{\em Det syder af fusel \ldots}
\end{vers}
\vfill
\begin{vers}
Min far han brændte i 67 år, 
og så blev han snuppet af tolden i går,
men mor har skaffet et nyt apparat   
og sat produktionen i gang i en fart.
{\em Det syder af fusel \ldots}
\end{vers}
\begin{vers}
Det hænder jeg selv ta'r en ordentlig syp,
mens jeg lytter til dråbernes sagte dryp-dryp.
Der er som musik fra et fint instrument,
men jeg ved, det er sprit på omkring 60 procent.
{\em Det syder af fusel \ldots}
\end{vers}
\begin{vers}
Hvis du har fået lyst til at smage det selv,
så kom kun til mig, helst når dag'n går på hæld,
slå 3 slag på ruden og stik mig en tier,
så får du en sjat af min livseliksir.
{\em Det syder af fusel \ldots}
\end{vers}
\laps
\end{sang}