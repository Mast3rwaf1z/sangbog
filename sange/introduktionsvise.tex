\begin{sang}{Introduktionsvise}{Melodi - Jeg er havren}
\spal2
\begin{vers}
Kære unge venner på Mat 1 !
Jeres nye studium blir {\em fedt},
I får kaffekrus og bajervom
- det skal I nu høre mere om!
\end{vers}
\begin{vers}
Kaffestuen er et hygsomt sted;
derfor går vi alle tit derned
for at mødes med det muntre sjak
læse og spille skak
\end{vers}
\begin{vers}
Fritiden blir brugt til endnu mer'
(indimellem skal man jo studer')
På Mat 1 der morer man sig med
at sende travle handelsrejsende af sted
\end{vers}
\begin{vers}
EDB er blevet populært
(det er vistnok heller ikke særlig svært)
Faktisk går der ikke længe før
man får set sin første ingeniør
\end{vers}
\vbox{}\vfill
\begin{vers}
Hvis det altså ikke helt slog klik
kan man jo ha` valgt at ta` fysik,
og da finder man den sande glæde
ovre på Pontoppidanstræde
\end{vers}
\begin{vers}
Til debat der skal der være tid \ldots
- så får faget meget mere BID \ldots
På Dat 3 er der filosofi
og talløse artikler
\end{vers}
\begin{vers}
Endnu mangler vi et dunkelt punkt
det er fessor, lektor og adjunkt
Een ting ved vel alle, hvor de bor :
På et kontor
\end{vers}
\begin{vers}
Der er BBK og PDV,
JS, AaN, AD og SLB,
PSE, MR og KGL
- resten kan I sikkert huske selv!
\end{vers}

\laps

\end{sang}