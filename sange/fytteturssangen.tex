\begin{sang}{Fytteturssangen}{Melodi - Julebal i Nisseland}
\spal2
\begin{vers}
Sikke mange klokken slår
Tretten slag - tiden går.
Gæt engang, min lille ven,
hvor vi nu skal hen.
\end{vers}
\begin{vers}
På fyttetur, på fyttetur til Møllegår'n
Hent lidt Classic, så suser vi af sted!
Nej vent nu lidt, og husk at ta' en lille tår
Milepælen, den må vi da ha' med!
På alle veje strømmer den glade femberflok.
Jeg tror, at jeg drømmer
Nej det er rigtigt nok.
I år skal vi til fytte-fytte-fytte-fytte-fyttetur,
Det' på tide at holde femberkur.
\end{vers}
\begin{vers}
Her er bamserind på fad
Bare spis - dejlig mad
Remoulade, fisk og grill
Hører der dertil!
\end{vers}
\begin{vers}
På fyttetur, på fyttetur til Møllegår'n
Ih du milde, nu danser de ballet
Jeg laver spjæt
Med fembers til den lyse morg'n
Hvis jeg ville, så fløj jeg li'så let.
Vi danser hele natten og laver hurlumhej
Vi blæser på Katten, så render den sin vej.
Spil op, musik,
til fule-fule-fule-fule-fulemik
Fnisehuen, den passer på en prik.
\end{vers}
\begin{vers}
Der er no'en som ikke vil
Tro der er fember til.
Hør, hvad jeg fortæller dem,
når jeg kommer hjem:
\end{vers}
\begin{vers}
I år var jeg på fyttetur til Møllegår'n
Åh - vi laved en masse god fskæg
Der er så sjovt til galefest i Møllegår'n
Jeg har travet en mil fra væg til væg.
De bedste venner har jeg
På fytteturen fået
Til næste oktober
Tar' jeg af sted igen
På fyttetur, på fytte-fytte-fytte-fytte-fyttetur.
Det' på tide at holde femberkur.
\end{vers}
\laps
\end{sang}