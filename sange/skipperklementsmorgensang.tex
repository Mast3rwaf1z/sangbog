\begin{sang}{Skipper Klements morgensang}{Ebbe Reich/Leif Varmark}
\spal2

\begin{vers}
Skærm jeres hus med grav og planke,
hvæs jeres leer snittende blanke;
frygt ikke Rantzaus sorte hær;
Silke skal vige for vadmelsklær.
\end{vers}

\begin{omkvaed}[b]
Bønder, tømrere, jydske knejte,
nu skal vi sejre i grevens fejde.
\end{omkvaed}

\begin{vers}
Bagved de riges glitrende hjælme
skjuler sig bange skælvende skælme.
Djævelen selv har gjort dem til pynt,
rustninger, skjolde og jorder og mønt.
\emph{Bønder, tømrere, jydske\dots}
\end{vers}

\begin{vers}
Mennesker bytter de ud for penge,
porten til frihed de låse og stænge,
piner og binder med arv og med gæld,
jorden som havde de skabt den selv.
\emph{Bønder, tømrere, jydske\dots}
\end{vers}
\begin{vers}
Rigdom si'r de er for de rige.
Biblen har noget andet at sige.
Fattig på jorden vandred Guds søn,
hented just ikke hos dem sin løn.
\emph{Bønder, tømrere, jydske\dots}
\end{vers}

\begin{vers}
Kirkens sorte skadesværme
skal sig for pigernes latter beskærme.
Spraglede herremændshaner på stand
møde nu skal den danske mand.
\emph{Bønder, tømrere, jydske\dots}
\end{vers}

\begin{vers}
Hanen har galet trende gange;
stå nu fast for vi er de mange.
Lad ikke fremtiden sige om os:
Rigdommen knægted os uden at slås.
\emph{Bønder, tømrere, jydske\dots}
\end{vers}
\laps
\end{sang}