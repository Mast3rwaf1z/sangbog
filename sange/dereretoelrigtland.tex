\begin{sang}{Der er et ølrigt lang}{Melodi - Der er et yndigt land}
	\spal 2
\begin{vers}
Der er et ølrigt land
Det står med nø d og næpp e
Blandt alt det p okkers vand
Blandt alt det p okkers vand
Det bugter sig i bar og kro
Det hedder gamle Danmark
Og her er øllen go'
Ja hver en øl er go'
\end{vers}
\begin{vers}
Her drak i fordums tid
Hver tillakked' kæmp e
Sin mjø d af fad med flid
Sin mjø d af fad med flid
Så prøved' han, ej uden mén
At finde sine b ene
Men faldt ved hver en sten
Ja hver en bautasten
\end{vers}
\vfill
\begin{vers}
Den øl endnu er skøn
Og gid den aldrig vælter
Lad baj'ren stå så grøn
Lad baj'ren stå så grøn
De ædle sorters skæne ø er
Med sutter, fulde svende
Og svimle danske mø er
Ja, svimle danske mø er
\end{vers}
\begin{vers}
Hil druk og fædreland
Hil hver en kølig bajer
Vi drikker dem vi kan
Vi drikker dem vi kan
Vort gamle Danmark - SKÅL! - b estå
Så længe øllet skummer
Og næsen den bli'r blå
Med rø de prikker på
\end{vers}
\laps
\end{sang}
