\begin{sang}{Til julebal i Nisseland}{}
\spal2
\begin{vers}
Sikke mange klokken slår ?
Tretten slag – tiden går.
Gæt engang, min lille ven,
hvor vi nu skal hen.
\end{vers}
\begin{vers}
Til julebal, til julebal i Nisseland
På med vanten, så suser vi afsted.
Nej vent nu lidt, du sjove lille nissemand
elefanten, den må jeg da ha´ med.
På alle veje strømmer den glade nisseflok.
Jeg tror, at jeg drømmer.
Nej det er rigtig nok.
I nat skal vi til jule-jule-jule-jule-julebal,
der er gilde i Nissekongens hal.
\end{vers}
\begin{vers}
Her er risengrød fra fad,
bare spis – dejlig mad
Tag en smørklat på din ske,
drys med julesne
\end{vers}
\begin{vers}
Til julebal, til julebal i Nisseland.
Ih du milde, nu danser de ballet.
Jeg laver spjæt så godt som nogen nisse kan.
Hvis jeg ville, så fløj jeg li´så let.
Vi danser hele natten og laver hurlumhej.
Vi blæser på katten, så render den sin vej.
Spil op, musik, til jule-jule-jule-jule-julemik.
Nissehuen, den passer på en prik.
\end{vers}
\vspace{10cm}
\begin{vers}
Der er no´en som ikke vil
tro der er nisser til.
Hør, hvad jeg fortæller dem,
når jeg kommer hjem:
\end{vers}
\begin{vers}
I nat var jeg til julebal i Nisseland.
Åh – vi laved en masse nisseskæg.
Der er så flot i kongens slot i Nisseland.
Jeg har travet en mil fra væg til væg.
De bedste venner har jeg blandt de små nissemænd.
I morgen, så ta´r jeg måske afsted igen.
Til julebal, til jule-jule-jule-jule-julebal.
Der er gilde i Nissekongens hal.
\end{vers}
\laps
\end{sang}