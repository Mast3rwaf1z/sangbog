\begin{sang}{Fru Jensen fra Tolne}{Melodi: Trad.}
\spal2
\begin{vers}
Fru Jensen fra Tolne havde ikke råd
men alligevel skulle hendes dreng dog lære noget.
Og hun spinked' og spared' en net lille sum,
så hendes Kresten ku' komme på seminarium.
{\em Kom så tu-ra-ja
ra-di-da-di-da
tu-ra ju-ra ju-ra ja.}
\end{vers}
\begin{vers}
Hun slæbte og sled både dag og nat,
og det, Kresten ikke fik, måtte hun betale i skat.
Men hver gang hun sendte sin søn en tier,
ja, så brugte han pengene til øl og pi'r.
\end{vers}
\begin{vers}
Så en dag kom posten til Fru Jensens hus
med et brev, som gjorde hende helt konfus.
Hun matte tørre næsen i sin lommeklud,
da hun læste, at Kresten nu var blevet smidt ud.
\end{vers}
\begin{vers}
{\tt Og så skrev Kresten et brev til sin mor.
Der stod:}
"Kære mor, det var ikke, fordi jeg var et dovent drog,
men de andre i klassen tog mig med ud på  sjov.
Hvad jeg ikke vidste før om sex og druk,
fik jeg meget hurtigt lært, jeg blev en værre buk".
\end{vers}
\begin{vers}
Med moralen i historien er det så som så:
Lær din dreng at drikke, som du lærer ham at gå,
så han kan ta' sig en skid og gå på røven med et brag
og alligevel gå i skole den næste dag.
\end{vers}
\laps
\end{sang}