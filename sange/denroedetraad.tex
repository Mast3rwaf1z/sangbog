\begin{sang}{Den røde tråd}{Shu-bi-dua}
\spal2

\begin{vers}
Hvad mon man er før man blir til
et stjerneskud -- et puslespil
en tanketorsk -- en gulerod
man må for fa'en ha' været no'ed.
\end{vers}

\begin{vers}
Moders barm var rød og hvid
til min far sa' hun: Glid.
Vi var en kæmpe børneflok
og der var aldrig penge nok.
\end{vers}

\begin{vers}
Otte år så var jeg klar
til at gå på kaffebar
min ryg så ud som strivret flæsk
for dagens ret var tørre tæsk.
\end{vers}

\begin{omkvaed}
Jeg voksed og koksed.
Jeg fandt en plads i solen.
I skolen lærte jeg at
skrive De med stort.
\end{omkvaed}

\begin{vers}
Men jeg ku' aldrig bli' til no'ed
trods mit store ordforråd
stod i kø på livets vej
mens andre overhalede mig.
\end{vers}

\begin{omkvaed}
Jeg voksed -- og koksed
jeg fandt min plads i skyggen
med ryggen op mod muren
hvorfor blev jeg født?
\end{omkvaed}

\begin{vers}
Og hvad mon der sker når man skal bort
når livets skjorte bli'r for kort
hvor mon den er den røde tråd
man må for fa'en da blive til no'ed.
\end{vers}
\laps
\end{sang}